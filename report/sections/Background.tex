\subsection{Location Data \&  Augmented Reality}
In modern smartphones, the GPS system is fast and accurate, especially for purposes of landmark geofencing, 
as utmost accuracy is not a requirement. The advancement in smartphone technologies also enable richer AR experiences, as higher quality 
experiences may be included with less concern with device processing power.\\
When these two technologies are combined, the experience is taken to another level, as the Augmented Reality experience can shift based on 
the device's real conditions and positioning. A level of immersion is reached, as users need to move and visit heritage sites in 
order to experience the Augmented Reality effects, and in return, they gain context and heritage information about the landmarks visited.   
\subsection{Google ARCore \& Unity Technologies}
Google ARCore framework greatly facilitates the implementation of AR experiences, without the need of reinventing everything from scratch. 
A simplified framework enables lower-cost projects, lower-qualified developers and faster integration of AR projects, 
providing a gateway to the mainstream acceptance of AR.\\
Unity 3D technologies combined with Google ARCore enhance the usability of ARCore, as developers are enabled to keep using existing, 
familiar tools to develop an Augmented Reality Experience. 
\subsection{Augmented Reality \& Tourism}
Augmented Reality has seen its success in the IT industry, as can be seen in []. However, according to [] in 2017, the potential of AR
in the tourism domain is still not explored enough, and thus the envelope is still to be pushed for further integration.