
\subsection{ARCore and Unity}
Google ARCore is a framework for the creation of AR experiences on android and IOS devices, and comes with great capabilities out of the box, and simplifies the workflow 
by removing the need of reinventing everything from scratch. 
A simplified framework enables lower-cost projects\cite{Huang2019}, lower-qualified developers and faster integration of AR projects, 
providing a gateway to the mainstream acceptance of AR.\\
Unity 3D technologies combined with Google ARCore enhance the usability of ARCore, as developers are enabled to keep using existing, 
familiar tools to develop an Augmented Reality Experience. As unity has been a longstanding game development engine, it also has a wide range of support, 
add-ons and a helpful community. Also enabling the development of different platforms such as desktop, web and mobile applications\cite{Greene2018}.
\subsection{Augmented Reality and Tourism}
Augmented Reality in the tourism domain is no new application works such as \cite{OZKUL2019} show promising results of the applications in the tourism domain.
Furthermore, \cite{JingenLiang2021} also presents an insight into the future of this domain application, and results show that the area is still blooming.


\subsection{Augmented Reality Enhancment}
% In modern smartphones, the GPS system is fast and accurate, especially for purposes of landmark geofencing, 
% as utmost accuracy is not a requirement. The advancement in smartphone technologies also enable richer AR experiences, as higher quality experiences may be included with less concern with 
% device processing power.\\
Augmented Reality encapsulates a wide are of technologies which ultimately aims at enhancing, changing and manipulating some aspect of reality as we know it. AR can also be combined with other novel technologies, 
\cite{Keckes2017} delves into how the implementation and variation of these technologies can change the resultant experience, including technologies such as location, routing, interactive views and more. 

When these diverse technologies are combined, we can change and improve the experience, tailoring it to specific applications, requirements and expectations. 
The whole experience can shift based on the device's real conditions, positioning and sensory data, allowing a deeper level of immersion to be reached. 
%  as users need to move and visit heritage sites in order to experience the Augmented Reality effects, and in return, they gain context and heritage information about the landmarks visited. \\
% \subsubsection{Geofencing}


% Augmented Reality has seen its success in the IT industry, as can be seen in []. However, according to [] in 2017, the potential of AR
% in the tourism domain is still not explored enough, meaning that there still is great potential for new research.