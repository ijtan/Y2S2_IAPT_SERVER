\subsection{The Server}
An api server was written in python, as due to the small scale of the application, this was ideal to meet the requirements whilst keeping the 
implementation simple enough. The server provides several endpoints which may be pinged, but only two particular endpoints are used. 
\subsubsection{Location Updates}
The server keeps track of a list of active devices, (though a unique identifier provided on requests), and their last known location.
The device regularly updates the sever wit lcoation infromation, and then requests nearby landmark data. 
The server loops through all landmarks, and calculates the distance between the device longitude and latitude positioning, and the landmark.
If the distance is below some threshold, it is added to a list of potnetially explorable landmarks, which is returned as a JSON response 
to the device. Each landmark entry also contains a geofence region, which when is larger than the distance, the landmarks is considered near 
the user, and the device can knmow that the AR mode can be enabled.
\subsection{Location Service}
On the device, the location is updated every second, with an intended accuracy of 0.1 metres 
(usually not met, but we try to be as accurate as possible).\\
The integrated unity function is used, and a listener is used to check for updates, which update the server,
and the landmarks list accordingly.
\subsection{Close Landmark Menu}
A scrolable list is used to show a list of the landmarks returned by the server. The title,
a short description, the distance and the bearing is shown for each entry, so the user may have some basic 
location info. Whenever the landmarks is very close, the entry become interactable, and the user can press it 
to enter AR mode near the landmark. 
\subsection{Augmented Reality Mode}
In this mode, the camera is shown to the user, and a 3D transluscnet floating window is spawned in 3D space.
The user may move around the panel, and observe the panel stays locked in 3D space. The panel shows
some deeper description about the near landmark. A carousel allows the userto see some images of the place
 (as provided through the API).  