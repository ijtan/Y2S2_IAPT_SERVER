\subsection{The Server}
An api server was written in python, as due to the small scale of the application, this was ideal to meet the requirements whilst keeping the 
implementation simple enough. The server provides several endpoints which may be pinged, but only two particular endpoints are used. 
\subsubsection{Location Updates}
The server keeps track of a list of active devices, (though a unique identifier provided on requests), and their last known location.
The device regularly updates the sever wit lcoation infromation, and then requests nearby landmark data. 
The server loops through all landmarks, and calculates the distance between the device longitude and latitude positioning, and the landmark.
If the distance is below some threshold, it is added to a list of potnetially explorable landmarks, which is returned as a JSON response 
to the device. Each landmark entry also contains a geofence region, which when is larger than the distance, the landmarks is considered near 
the user, and the device can knmow that the AR mode can be enabled.
\subsection{Location Service}
On the device, the location is updated every second, with an intended accuracy of 0.1 metres 
(usually not met, but we try to be as accurate as possible).\\
The integrated unity function is used, and a listener is used to check for updates, which update the server,
and the landmarks list accordingly.
\subsection{Close Landmark Menu}
A scrolable list is used to show a list of the landmarks returned by the server. The title,
a short description, the distance and the bearing is shown for each entry, so the user may have some basic 
location info. Whenever the landmarks is very close, the entry become interactable, and the user can press it 
to enter AR mode near the landmark. 
\subsection{Augmented Reality Mode}
In this mode, the camera is shown to the user, and a 3D transluscnet floating window is spawned in 3D space.
The user may move around the panel, and observe the panel stays locked in 3D space. The panel shows
some deeper description about the near landmark. A carousel allows the userto see some images of the place
 (as provided through the API).  

 \subsection{Other AR Techniques}
 A couple of other Augmented Reality techniques were explored during the development of this app. 
 These tecniques were implemented and worked really well as sandalone, yet when combining the features 
 the standalone applications do not have an yuses, and thus there is no way of using them. 
 \subsubsection{Plane detection}
 Although raw plane detection was not used in the final version, under the hood google uses it to keep the 
 floating infromation panel in place. Through goggle's AR Core it is made possible to detect vertical 
 and horizontal planes, to which other game opbjects can be anchored to!

 In an example, plane detection was used to find a stable surface, and when the user clicks on a 
 plane, a 3D Game model is spawned in place, and anchored to the plabe. The user is also to walk around 
 in the room, whilst the objects stays anchored to plane! 

 \subsubsection{Image Recognition \& Augmented Images}
 Image Recognition was also a really interesting feature to use. In this projects case, a quick 
 database manager was created in which a list of images could be inserted. And actions would be taken 
 according to the image detected!
 
In an example, an image of the earth was used as a key, and when this image is detected, a 3D spinning 
globe would be overlayed on it, where the user is able to go around the globe!
This may have been abnle to implemented in the app, yet as the landmark menu and the ARMode switching 
works, It did not have much room to be used. (As in the near landmark menu, there is 
not access to the camera), and the user may only use AR Mode when near a landmark. However, this 
feature is also fully working, and may easily be implemented if a better use is identified. 

