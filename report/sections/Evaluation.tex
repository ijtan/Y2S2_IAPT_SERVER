Due to the domain of this project and the technologies adopted, it was not as straightforward to 
test the raw performance of the system. As there is no performance metric or such that can be 
analysed and compares to other instances. \[https://ieeexplore-ieee-org.ejournals.um.edu.mt/stamp/stamp.jsp?tp=\&arnumber=8120326\] 
\subsection{Testing Performed}

\subsubsection{Landmarks AR Information Panel}
After finalizing the system (application and server), testing took place in the 
real locations of the landmarks. Through this on-site testing an idea of how the application 
performs was obtained.\\
Since the floating information panel stays in place using plane detection and sensory data, care 
was given to push the limits of these technologies. Plane detection struggles when there are 
no close discernable attributes (the floor has a regular pattern for example), which the tests 
have shown that also effects the stability of the position of the panel. 
\subsubsection{Augmented Images}
As augmented images were implemented (yet mever used) some basic testing was also involved, 
which turned out to be quite succesful. The image was recognized from different angles and 
light settings whilst tracking also was really responsive to even moving the image.  
\\
Further testing is shown in the video, as it's the easiest way to show off the experience.
\subsection{Potential Testing}
Three key points mentioned in \[https://ieeexplore-ieee-org.ejournals.um.edu.mt/stamp/stamp.jsp?tp=\&arnumber=8120326\] 
may be used to give a more formal idea of the performance of such sustems. 
\subsubsection{Independant Variables}
The variation of independant variables can be used as a perfomance metric and help analyse 
the experience of the user, these variables include things which are not varied by the user during 
the test. Examples of such variables include the device size and the field of view of a device's 
camera.\\
\subsubsection{dependant Variavles}
These variables give a robust metric of how the users react to the 
application presented, such as the number of attempts taken to carry out an action.
Yet this is very application specific, and does not allow for comparison between other systems, as 
tasks in an application usually are specific to it.
\\

\subsubsection{Questionnaires}
Questionnaires were mentioned as another performance metric, which allows a subjective metric. 
In applciations such as the tourism domain this is of utmost importance, as it follows that 
from the user-centric nature, the ultimate goal is the users' experience.  
