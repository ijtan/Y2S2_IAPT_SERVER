
\subsection{Test Analysis}

\subsubsection{Landmark Menu}
The solution as an application was very stable in general, since unity was used, any changes can be done directly through the unity editor, and the benefits of being a 
unity-based application are retained. The aim to provide some sense of direction to the user was also satisfied, as when the GPS connection was established, distance 
and bearing data was accurate. When the GPS connection was unstable, the bearing direction and distance were a little off, yet immediately normalized when the connection was restored. 
\subsubsection{AR Mode}
In general, when discernable plane features are present, the panel stays fixed in place as intended.
Unfortunately, plane detection by design struggles when there are no close discernable attributes (the floor has a regular pattern for example) and there is not much which can be done to 
counteract this issue. The tests performed have also shown that in rare cases, this affects the stability of the position of the panel. However, this does not occur very often, and the 
general experience rarely suffers.
\subsubsection{Server Communication}
Through using the application, the server was also being tested. The solution developed was very responsive, and provided very quick and usable results. 
Distance calculation and geofencing also worked punctually, and the device's locations were updated in real-time.\\
As expandability was also in mind, it is really easy to add, remove and change landmarks, as a simple JSON file is provided, where every bit of detail can be changed.\\
Due to the centralised nature of the application, an internet connection needs to be maintained for most features to work, 
as all landmark information is obtained from the server.
\subsubsection{Critical Analysis}
The few tests performed show that the technologies used are very applicable to the aims and objectives. The strengths of the application match up with the strengths of AR as a concept, 
providing a deep sense of immersion whilst also being informative. On the other hand, the weaknesses of the application are also in sync with the limitations of the concepts by design. Particularly, the application suffers where a stable 
plane is not detected and a plane with lower certainty has to be used, which may cause the information panel to shift positions unexpectedly. 

\subsection{Potential Further Testing}
Three key points mentioned in \cite{Samini2017} have great potential in giving a more formal understanding of the performance of such systems. 
\subsubsection{Variation of Independant Variables}
The feedback provided after variation of independent variables can be used as a performance metric and help analyse the experience of the user, these variables include things that are not 
varied by the user during the test. Examples of such variables include the device size and the field of view of a device's 
camera, which can help conclude the applicability of the technologies adopted with the devices used.
\subsubsection{Variation of Dependent Variables}
The variation of these variables gives a robust metric of how the users react to the 
application presented, such as the number of attempts taken to carry out an action.
Yet this is very application-specific and does not allow much room for comparison between other systems, as 
tasks in an application usually are specific to it.

\subsubsection{Questionnaires}
Questionnaires were mentioned as another performance metric, in contrast to the other tests, this metric is purely subjective. 
In applications such as the tourism domain, this is of utmost importance, as it follows that 
from the user-centric nature, the ultimate goal is to maximise the users' resultant experience, and thus feedback directly from the users may be key.  

% \subsection{Future Works}
% The system was developed with expandability in mind, and thus, it is really easy to adapt the system into different uses cases.
\subsection{Possible Improvements}
\subsubsection{Augmented Images}
Ideally, the augmented imaging feature mentioned in appendix 1 is implemented as another landmark type, as it would allow for a wider range of applications and immersion.
This application would also be applicable in places such as inside museums where for example it would enhance a painting, or have some animation overlayed on it.
\noindent
Another possible improvement is to give better direction towards landmarks, the current application simply gives baring direction and straight line distance between points,
 without any consideration for the streets/obstacles.
\subsection{Future Work}
During development, it was made sure to keep the system as open as possible. Through the 
centralised API, unity engine and other technologies used, the system is meant to be dynamic and 
expandable.\\
With minimal effort, it can be easily be adapted to different uses such as an AR game, 
using google APIs for international standardized locations or even showing 3D models in AR mode. 

